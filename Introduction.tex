
The interpretation of searches for Dark Matter (DM) (or any other LHC
physics result) requires that one assumes a model leading to the
signal under consideration. This is necessary to compare searches across channels, searches at other center-of-mass energies or at other collider
experiments. The ATLAS and CMS experiments at the LHC coordinated
in 2015 a joint forum to address this issue, in collaboration with
theorists. This ATLAS/CMS DM Forum produced 
a report~\cite{Abercrombie:2015wmb}, providing a first set of concrete simplified DM models used by
ATLAS and CMS to interpret their searches for missing transverse
energy~(MET) signatures.

At the end of the DM forum's activities, a formal LHC Dark Matter WG
(LHCDMWG) was created, to continue the discussion and harmonisation of the way
in which the LHC DM results are interpreted, reported and compared to those of
other experimental approaches. 


This document provides the LHCDMWG recommendations on how to present the LHC
search results involving the $s$-channel models considered
in~\cite{Abercrombie:2015wmb} and how to compare these results to
those of direct (DD) and indirect detection (ID) experiments. This
document is the result of the discussions that took place during the
first public meeting of the LHCDMWG~\cite{LHCDMWGWorkshop}, and it is
intended to provide a template for the presentation of the LHC results
at the winter conferences in 2016. It reflects the feedback obtained from
the participants and in subsequent iterations with members of the
experiments and of the theory community and it is based on work
described recently
in~\cite{Buchmueller:2014yoa,Abdallah:2014hon,Malik:2014ggr,Buckley:2014fba,Harris:2014hga,Haisch:2015ioa,Abdallah:2015ter}. For
earlier articles discussing aspects of simplified $s$-channel DM
models, see
also~\cite{Petriello:2008pu,Gershtein:2008bf,Dudas:2009uq,Bai:2010hh,Fox:2011pm,Goodman:2011jq,An:2012va,Frandsen:2012rk,Dreiner:2013vla,Cotta:2013jna,Buchmueller:2013dya,Abdullah:2014lla}.

The relevant details of simplified DM models involving vector,
axial-vector, scalar and pseudo-scalar $s$-channel mediators are first
reviewed in Section~\ref{sec:models}.
Section~\ref{sec:colliderresults} presents a recommendation for the
primary treatment of LHC DM bounds and introduces all of the basic
assumptions entering the approach.
Section~\ref{sec:comparisontonon-colliderresults} describes a
well-defined translation procedure, including all relevant formulas
and corresponding references, that allows for meaningful and fair
comparisons with the limits obtained by DD and ID experiments.








